\documentclass[]{article}

%opening
\title{Implementing of a simple Heat-Convection-Diffusion Model}
\author{Florian Nikos Kitzka}

\begin{document}

\maketitle

\begin{abstract}
The aim of this document is to provide a simple guide how to create and implement a model for a heat-convection-diffusion problem. Our approach is to start from underlying physics to pose the model-equations and then to explain difficulties arising during implementation.
The full implemented result is provided by an external link, which deems to encourage the reader to extend and play with given implementation.
We assume basic knowledge of Analysis and Linear Algebra as well as some understanding of 
fluid dynamics. Chapters are written quite independent of each other so that the reader
can skip parts in which no interest exist. 
\end{abstract}

\section{Simulation Target}
We consider a room of cubical shape.\\
We assume all walls except the bottom to allow transfer of air (open walls).\\
The bottom is assumed to be heated by the sun-light through radiation.\\
Our goal is to simulate how this heat is being transferred through the room over time.
TODO picture

\section{Creation of Model}
There are two main mechanism of heat transfer in nature. One is diffusion the other is convection.

\section{Numerical Scheme}
In order to solve the model equation we are going to use the so called
Finite Difference Method.

\subsection{Derivative Approximations}

\subsection{Stability}
One of the most important things concerning numerical solution of approximation
schemes is to ensure stability.
Actually one could argue that since we have justified all our derivative approximations by Taylor-expansions we are ready and can blindly implement the discrete equations into 
a computer-system. This approach though turns out too naive. Remember that any computer system always produces rounding errors. Although these errors are very small they can
sum-up drastically when we have to repeat calculations very often.
Our scheme is of the form 
\begin{equation}
u_{n+1}=A(u_{n})
\end{equation}
where $A$ is some linear function. 

 

\subsection{Operator and Term Splitting}

\subsection{Final Schema}

\section{Implementation}

\section{Further Reading}
This article intends to give an introduction into all treated areas.
There are many good books or only tutorials about numerical treatment of partial
differential equations. Or if you are more specialized on fluid dynamics you will
many good accounts on this field too. In case your are more interested in the 
implementation side, be encouraged to clone the entire project from https://github.com/applied-math-coding/basic-diffusion-transport and to extend or play around.
\end{document}
